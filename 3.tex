\documentclass{article}
\usepackage{amsmath}
\usepackage{amssymb}
\usepackage[margin=1in]{geometry}
\usepackage{array} 
\newcommand{\tuple}[1]{\ensuremath{\left\langle #1 \right\rangle}}


\begin{document}

\textbf{Exercice 3}

\vspace{0.5cm}
1.
$$ C = \{ [0]_{mod{4}},[1]_{mod{4}}, [2]_{mod{4}},[3]_{mod{4}} \}$$

$$f(0) = [0]_{mod{4}} \hspace{1cm} \text{car} \hspace{1cm} 0 = 0 \mod{4} \hspace{1cm} \tuple{ 0 = 0 + 4\cdot0 }$$

$$f(42) = [2]_{mod{4}} \hspace{1cm} \text{car} \hspace{1cm} 42 = 2 \mod{4} \hspace{1cm} \tuple{ 42 = 2 + 4\cdot10 }$$

$$f(43) = [3]_{mod{4}} \hspace{1cm} \text{car} \hspace{1cm} 43 = 3 \mod{4} \hspace{1cm} \tuple{ 43 = 3 + 4\cdot10 }$$

2. Oui l'image de $f$ couvre les classes de $\theta$ car par définition les classes partitionnent $S$ et leur union doivent couvrir $S$.

3. Non, ce n’est pas toujours vrai, mais cela peut l’être. Par exemple $\mathbb{N}$ avec la définition habituelle de l'égalité comme sa relation d'équivalence est injective. Mais par exemple le modulo 4 comme ci-dessus n'est pas injectif. l'image de $f$ couvre les classes modulo 4 plusieurs fois. En effet,

$$ f(0) = [0]_{mod{4}} = f(4) \hspace{1cm} \tuple{ 4 = 0 \mod{4} } $$

et pourtant $ 0 \neq 4 $ dans $\mathbb{N}$





\end{document}
