
% !TeX spellcheck = fr_CA
\documentclass[11pt]{article}

\newcommand{\ENONCE}[1]{#1}
\newcommand{\REMISE}[1]{}
 \usepackage[T1]{fontenc}
\frenchspacing
%\usepackage[francais]{babel}  % genere les trucs automatique en francais
\usepackage{enumerate}

\usepackage{hyperref}
\usepackage[canadien]{babel}
\usepackage[utf8]{inputenc}
\usepackage[all]{xy}
\usepackage{multicol, array}

\usepackage[top=25mm, bottom=25mm, left=25mm, right=25mm]{geometry}

\usepackage{amssymb}
\usepackage{amsmath,wasysym}
\usepackage{epigraph}
\usepackage{subfig}
\usepackage{framed}
\usepackage{graphicx}
\usepackage{color}


%%% Couleurs du texte %%%
\newcommand{\rouge}[1]{\textcolor{red}{#1}}
\newcommand{\bleu}[1]{\textcolor{blue}{#1}}
\newcommand{\blanc}[1]{\textcolor{white}{#1}}

%%% Valeurs booléennes %%%
\newcommand{\vrai}{\mbox{\tt vrai}}
\newcommand{\faux}{\mbox{\tt faux}}
\newcommand{\V}{\mbox{\tt v}}
\newcommand{\F}{\mbox{\tt f}}

%%% Opérateur booléens %%%
\newcommand{\non}{\neg}                  % Opérareur de négation
\newcommand{\et}{\wedge}                 % Opérareur de conjonction
\newcommand{\ou}{\vee}                   % Opérareur de disjonction
\newcommand{\implique}{\Rightarrow}      % Opérareur d'implication
\newcommand{\impliqueinv}{\Leftarrow}    % Opérareur d'implication inverse
\newcommand{\ssi}{\Leftrightarrow}       % Opérareur si et seulement si

%%% Opérateurs ensemblistes %%%
\newcommand{\dans}{\in}                  % Opérateur d'appartenance
\newcommand{\inclus}{\subseteq}          % Opérareur d'inclusion
\newcommand{\strictinclus}{\subset}      % Opérareur d'inclusion stricte
\newcommand{\inclusinv}{\supseteq}       % Opérareur d'inclusion inverse
\newcommand{\strictinclusinv}{\supset}   % Opérareur d'inclusion stricte inverse

\newcommand{\nondans}{\notin}              % Négation de l'opérateur d'appartenance
\newcommand{\noninclus}{\not\subseteq}     % Opérareur d'inclusion
\newcommand{\nonstrictinclus}{\not\subset} % Opérareur d'inclusion stricte

\newcommand{\inter}{\cap}                 % Opérateur d'intersection
\newcommand{\union}{\cup}                 % Opérateur d'intersection
\newcommand{\comp}{{\mbox{\footnotesize $c$}}} % Complément
\newcommand{\moins}{\setminus}

%%% Ensembles fréquemments utilisés %%%
\newcommand{\ensembleVide}{\emptyset}  % L'ensemble vide
\newcommand{\ensembleU}{{\mathbf{U}}}  % L'ensemble universel
\newcommand{\ensembleB}{{\mathbb{B}}}  % L'ensemble des nombres booléens
\newcommand{\ensembleN}{{\mathbb{N}}}  % L'ensemble des nombres naturels
\newcommand{\ensembleZ}{{\mathbb{Z}}}  % L'ensemble des nombres relatifs
\newcommand{\ensembleR}{\mathbb{R}}    % L'ensemble des nombres reels
\newcommand{\ensembleQ}{\mathbb{Q}}    % L'ensemble des nombres rationnels

%%% Relations  %%%
\newcommand{\compose}{\circ}            % Opérateur de composition
\newcommand{\croix}{\times}             % Opérateur de produit cartésien
\newcommand{\Identite}{{\mathbf{I}}}    % Relation Identitée
\newcommand{\Domaine}{{\mbox{\rm Dom}}} % Domaine
\newcommand{\Image}{{\mbox{\rm Im}}}    % Image
\newcommand{\tuple}[1]{\ensuremath{\left\langle #1 \right\rangle}}  % n-tuple
\newcommand{\Rcal}{{\mathcal R}}       % «R» caligraphique
\newcommand{\Lcal}{{\mathcal L}}       % «L» caligraphique
\newcommand{\floor}[1]{\lfloor #1 \rfloor}       % 
\newcommand{\ceill}[1]{\lceil #1 \rceil}       % 

%%% Démonstrations %%%
\newcommand{\cqfd}{\blanc{.}\\[-2mm]\mbox{}\hfill {\bf C.Q.F.D.}\\} % C.Q.F.D
\newcommand{\EXPLICATION}[1]{ \blanc{.}\hfill $\left\langle \mbox{ \small\it #1 } \right\rangle$ }
\newcommand{\EXPLIQUE}[1]{ \phantom{.}\hfill $\left\langle \mbox{ \small\it #1 } \right\rangle$ }
\newcommand{\MEXPLIQUE}[1]{\mbox{\EXPLIQUE{#1}}}
\newcommand{\Justif}[2]{\left\langle\mbox{ \small \it \begin{minipage}{#1cm}#2 \end{minipage} }\right\rangle}


\newcommand{\suite}[2]{\tuple{#1_n}_{n\in#2}}
\newcommand{\dsum}{\displaystyle\sum}

%%% Divers %%%
\newcommand{\eqdef}{\overset{{\mbox{\rm\tiny def}}}{=}} % Symbole de définition
\newcounter{exercice}\newcommand{\exercice}{ \bigskip \addtocounter{exercice}{1}\noindent \textbf{Exercice \theexercice}\\}
\newcommand{\reponse}[1]{\REMISE{\vspace{.5cm}\noindent\textbf{Réponse : } #1 \newpage}}
\REMISE{\renewcommand{\ENONCE}[1]{}}
\newcommand{\Dom}{{\mbox{\rm Dom}}}
\newcommand{\Img}{{\mbox{\rm Im}}}

\renewcommand{\mod}{\textrm{\,mod\,}}
\newcommand{\solution}[1]{}%\\\bleu{#1}}


\begin{document}
% ---------------------------------- DéBUT DES MODIFICATIONS À FAIRE ---------------------------
% Décommenter la ligne suivante pour compiler en mode ``Remise de travail''
%\renewcommand{\REMISE}[1]{#1} 

\REMISE{
\noindent
\begin{tabular}{l}
% Modifiez les noms etc  ---------------------------------------------------------------------
% Les noms ne sont pas nécessaires, surtout si vous pensez que nous utiliserons votre solution pour le solutionnaire général ;)
Nom1 B-GLO ou B-IFT\\
Nom2 B-GLO ou B-IFT
\end{tabular}
}
\noindent
 {\LARGE \textbf{DEVOIR 3} } \hfill {\Large\bf MAT-1919:  Automne 2024}\\[1mm]
%\emph{À remettre \emph{le } à 23h55 au plus tard}

\begin{center}
 \textcolor{blue}{Répondre à tous les numéros qui ne sont pas marqués facultatifs, car seulement certains seront corrigés. Pour les autres, vous aurez quelques points pour les avoir faits, mais aucune pénalité.}
\end{center}

\ENONCE{
\noindent
\emph{
Toutes les consignes suivantes seront considérées dans la note :
\begin{itemize}%\addtolength{\itemsep}{-4pt}
\item[$\bullet$] Enregistrez votre équipe \rouge{ENCORE} (1 à 3 étudiants) avant la date limite. % même si le ciel est bleu. 
%
\item[$\bullet$]  Remise :  {{\textbf{un} fichier pdf \textbf{par exercice} (un seul)}, chacun étant  identifié par son numéro comme \bleu{dernier} caractère} (exemple: \emph{1.pdf, 2.pdf}). Pas de zip, pas de jpg. 
\item[$\bullet$]  Soignez la \textbf{lisibilité} et l'orthographe. \bleu{Remettez un travail professionnel, même écrit à la main! }Les photos sont souvent  de \textbf{piètre qualité} (prenez Office Lens).  
\item[$\bullet$]  Retard :  les 2 premières heures non pénalisées; ensuite -1\% par heure de retard.
\end{itemize}
\noindent .
\\\bleu{Travaillez en équipe si possible,  faites  tous les numéros, il en va de votre préparation à l'examen final.}
}}
\\
Certains numéros sont marqués comme \textbf{facultatifs}. Ils seront commentés, c'est-à-dire corrigés,  mais vous n'obtiendrez pas de points, ni de pénalité si vous ne les faites pas. Ces numéros restent importants pour votre \textbf{préparation à l'examen final}.

\exercice 
Démontrez la dénombrabilité  de l'ensemble $\ensembleZ_{<0}\times \ensembleN$ sans construire de bijection à partir de $\ensembleN$, mais plutôt  en utilisant des théorèmes des notes de cours. 



\exercice\vspace{-3mm}
\begin{enumerate}[A)]%\addtolength{\itemsep}{-4pt}
\item  Tracez le diagramme de Hasse des relations ci-dessous, si c'est possible.  Sinon, 
\begin{itemize}
\item donnez la propriété qui l'empêche et pourquoi elle n'est pas vérifiée; s'il y en a plus d'une, dites-les toutes.
\item  si la relation est une relation d'équivalence, donnez ses classes d'équivalence, de deux façons: 
\begin{itemize}%\addtolength{\itemsep}{-4pt}
\item 1) en les écrivant en extension (comme ensembles) 
\item 2) à l'aide d'un représentant de la classe (pour chaque classe).
\end{itemize}
\item si la relation n'est pas une relation d'équivalence, donnez la propriété qui l'empêche et pourquoi elle n'est pas vérifiée; s'il y en a plus d'une, dites-les toutes.
\end{itemize}


\begin{enumerate}[a)]%\addtolength{\itemsep}{-4pt}
\item  La relation sur $\{0,1,2,3,4,5\}$  suivante 
${\cal R} = I_{\{0,1,2,3,4,5\}} \cup\{  \tuple{0,0},\tuple{1,4}, \tuple{3,2},\tuple{4,5}\}$
\item La fermeture transitive de la relation étudiée en a)
\item  L'union de la relation considérée en b) avec son inverse.
\item  La clôture transitive de la relation étudiée en c)
\end{enumerate}\end{enumerate}
\noindent
\emph{Pour B) répondez aux mêmes questions qu'en A)  mais ne dessinez pas le diagramme de Hasse si c'est un ordre partiel et dites  si c'est un ordre total; d'autre part, si c'est une relation d'équivalence, plutôt que de donner toutes ses classes d'équivalences, décrivez-en une et dites combien il y en a. Justifiez le tout informellement.}
\begin{enumerate}[B)]
\item \textbf{(Facultatif)}  Sur les mots binaires de longueur 3, considérez la fermeture réflexive\footnote{Si ${\cal R}\subseteq S^2$ est une relation, sa fermeture/clôture réflexive est ${\cal R}\cup {\cal R}^0$.} de la relation ${\cal L}$: «~a un plus petit nombre de 1  (ou égal) que », par exemple, $100\,{\cal L}\, 011$, $010\,{\cal L}\, 010$, etc. 
\\ \emph{Réflexion (réponse non obligatoire):} Le fait de prendre la \emph{fermeture réflexive} dans cette question, qu'est-ce que ça change, et quel est l'avantage que cette notion de \emph{fermeture réflexive} soit définie?
\end{enumerate}
\newpage

\exercice 
Soit $\theta$ une relation d'équivalence sur un ensemble $S$. Considérons $C = \{[s]_{\theta} \mid s\in S\}$ et $f: S\to C$ la fonction qui envoie un élément $s \in S$ sur $[s]_{\theta}$. 
\begin{enumerate}%\addtolength{\itemsep}{-4pt}
\item Considérons la relation d'équivalence $\theta:=~\simeq_{\mod 4}$ sur $S:=\mathbb{N}$, déjà vue en cours. Donnez $C$ pour cette relation (en utilisant la notation concise pour les classes d'équivalences que vous voulez écrire). Donnez les valeurs de $f(0)$, $f(42)$, et $f(43)$.
\end{enumerate}
\textbf{(Facultatif)} Reprenons $S$, $T$ et $\theta$  quelconques.
\begin{enumerate}
\item [2.]
Informellement, dites si $f$ est surjective (toujours? parfois? dans quels cas?)
\item[3.]
Informellement, dites si  $f$ est injective  (toujours? parfois? dans quels cas?)
\end{enumerate}



\exercice
 Soit   la suite $\tuple{D(n)}_{n\in\{2^i\mid i\in \mathbb{N}\}}$ définie par la règle de récurrence suivante:
 $$\begin{cases}
D(1)=7\\
D(n) = 2 * D( \frac{n}{5}) + n \quad \forall n \in \{2^i\mid i\in \mathbb{N}^*\}  
\end{cases}$$
Cette suite a une forme typique de calcul de temps d'exécution d'un algorithme! vous en verrez de semblables en IFT-3001.

\begin{enumerate}
 \item \textbf{(Facultatif)} Donnez les cinq premiers termes de la suite (pour pouvoir vérifier votre réponse à la prochaine sous-question!)
 \item En utilisant la méthode des substitutions à rebours, déduisez le terme général de la suite.  (Une méthode complémentaire devra être utilisée à la fin.) Ne laissez pas $n$ en exposant dans votre réponse s'il y a  moyen de le simplifier pour qu'il ne s'y retrouve plus!
\item 
Au début de la section 2.6 des notes de cours, on affirme que la méthode d'approximation par une intégrale est typiquement utilisée après avoir effectué une substitution à rebours.  Pourrions-nous/devrions-nous utiliser la méthode d'approximation par une intégrale pour borner la valeur de cette suite? Justifiez brièvement (une phrase suffit).

\end{enumerate}

\exercice
Trouvez le terme général avec Wolfram Alpha de la suite $\langle b_n\rangle_{n\in\ensembleN}$ dont la définition par récurrence est\\
$$\left\{
\begin{array}{cclr}
  b_0 & = & 4 &  \\
  b_n & = &  3 \,b_{n-1}\,+\,n\,+4 & \, \forall n\in\ensembleN^\ast 
  %format wolframalpha : b(0)=4, b(n)= 3b(n-1) +n +4
\end{array}\right.$$
%
\\
et démontrez que cette réponse est la bonne en utilisant la technique proposée dans le chapitre 2. 
 
\reponse{
}


\exercice
En n'utilisant ni la technique de substitution à rebours, ni la démonstration par induction, mais plutôt un théorème du manuel, trouvez le terme général\footnote{Soyez futés, confrontez votre réponse finale avec quelques valeurs $b_0$, $b_1$, $b_2$, $b_3$, $b_4$ calculées directement!} de la suite $\langle b_n\rangle_{n\in\ensembleN}$ dont la
définition
par récurrence est
$$\left\{
\begin{array}{cclr}
  b_0 & = & 4 &  \\
  b_n & = &  b_{n-1}\,+\,2n\,+ 4 & \, \forall n\in\ensembleN^\ast 
\end{array}%
\right.$$

\reponse{
}
\exercice
\rouge{Exercice sur les graphes, à venir. CRÉEZ VOTRE ÉQUIPE EN ATTENDANT!}

\reponse{
}

\exercice
\textbf{(Facultatif)} En utilisant la méthode par série génératrice, trouvez le terme général de la suite suivante:
$\\ \left\{
\begin{array}{cclr}
  b_0 & = & 2 \,\, &  \\
  b_n & = & 2\,b_{n-1}\,+\,3n\,-\,4& \, \forall n\in\ensembleN^\ast \\
  %
\end{array}%
\right.$  \\[1mm]
 
\reponse{
}



\vfill
\vfill\vfill\vfill\vfill\vfill\vfill\vfill

\end{document}

