\documentclass{article}
\usepackage{amsmath}
\usepackage{amssymb}
\usepackage[margin=1in]{geometry}
\usepackage{array} 
\newcommand{\tuple}[1]{\ensuremath{\left\langle #1 \right\rangle}}


\begin{document}

\textbf{Exercice 6}

\vspace{0.5cm}

Commençons par calculer $b_n - b_{n-1} = \tuple{2n + 4}_{n \in \mathbb{N}^{*}} = \tuple{f(n)}_{n \in \mathbb{N}^{*}}$ et calculons ensuite $f(0) = 2 \cdot 0 + 4 = 4 = b_0$. Ainsi, $f(n) = 2n + 4$ est une suite arithmétique avec un premier terme $4$ et une différence $2$. Si nous appliquons ensuite le Théorème 2.4.6, nous obtenons

\begin{align*}
b_n &= \frac{(4 + 4 + 2n)(n+1)}{2} 
&\tuple{ \text{par le théorème 2.4.6} } \\
&= \frac{2n^2 + 10n + 8}{2} 
&\tuple{ \text{arithmétique} } \\
&= n^2 + 5n + 4 
&\tuple{ \text{arithmétique} }
\end{align*}

Calculons maintenant quelques termes à l'aide de la relation et de la forme géneral. 

\begin{align*}
b_1 &= 4 + 2 \cdot 1 + 4 = 10 
&b_1 &= 1^2 + 5 \cdot 1 + 4 = 10 \\
b_2 &= 10 + 2 \cdot 2 + 4 = 18 
&b_2 &= 2^2 + 5 \cdot 2 + 4 = 18 \\
b_3 &= 18 + 2 \cdot 3 + 4 = 28 
&b_3 &= 3^2 + 5 \cdot 3 + 4 = 28 \\
b_4 &= 28 + 2 \cdot 4 + 4 = 40 
&b_4 &= 4^2 + 5 \cdot 4 + 4 = 40
\end{align*}

Ainsi, notre forme général $b_n = n^2 + 5n + 4$ semble être correcte.



\end{document}
