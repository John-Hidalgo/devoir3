\documentclass{article}
\usepackage{amsmath}
\usepackage{amssymb}
\usepackage[margin=1in]{geometry}
\usepackage{array} 
\newcommand{\tuple}[1]{\ensuremath{\left\langle #1 \right\rangle}}


\begin{document}

\textbf{Exercice 5}

\vspace{0.5cm}
On constate que $\frac{27}{4}3^n - \frac{1}{2}n - \frac{11}{4}$ est le terme générale de la suite $\tuple{b_n}_{n \in \mathbb{N}}$.

\vspace{0.5cm}

\textbf{Démonstration} On le démontrera par induction. Et pour ce faire, Définissons le prédicat.

$$P(n): b_n = \frac{27}{4}3^n - \frac{1}{2}n - \frac{11}{4} $$ 

si l'on peut démontrer 

$$P(0) \land (\forall n \in \mathbb{N}^{*} \mid P(n-1) \implies P(n)),$$

alors le principe d'induction mathématique impliquera $(\forall n \in \mathbb{N} \mid P(n))$. À cette fin,
\vspace{0.5cm}

\textbf{Case de base: montrons} $P(0). \hspace{4cm} \tuple{ \text{ montrons } b_0 = \frac{27}{4}3^0 - \frac{1}{2}\cdot 0 - \frac{11}{4}}$

Et en effet,

$$ \frac{27}{4}3^0 - \frac{1}{2}\cdot 0 - \frac{11}{4} = 4 = b_0$$

Donc on a bien que $P(0)$ est vrai.

\vspace{0.5cm}
\textbf{Étape d'induction: montrons} $(\forall n \in \mathbb{N}^{*} \mid P(n-1) \implies P(n))$

Soit $n \in \mathbb{N}^{*}$ et supposons que $P(n-1)$ est vrai. Que,

$$ \frac{27}{4}3^{n-1} - \frac{1}{2}(n-1) - \frac{11}{4} \hspace{5cm} \tuple{ \text{ et montrons } P(n)} $$

Puis par definition, 

$$ b_n = 3b_{n-1} + n + 4 $$
$$ = 3(\frac{27}{4}3^{n-1} - \frac{1}{2}(n-1) - \frac{11}{4})+ n + 4 \hspace{2cm} \tuple{ \text{ car } P(n-1) \text{ est vrai }}$$
$$ = \frac{27}{4}\cdot3\cdot3^{n-1} - \frac{3}{2}n + \frac{3}{2} - \frac{33}{4} + n + 4 \hspace{2cm} \tuple{ \text{ Simplification algébrique. } }$$
$$ = \frac{27}{4}3^n - \frac{1}{2}n + \frac{3}{2} - \frac{33}{4} + 4 \hspace{2cm} \tuple{ \text{ Simplification algébrique. } }$$
$$ = \frac{27}{4}3^n - \frac{1}{2}n + \frac{11}{2} - \frac{33}{4} \hspace{2cm} \tuple{ \text{ Simplification algébrique. } }$$
$$ = \frac{27}{4}3^n - \frac{1}{2}n - \frac{11}{4} \hspace{2cm} \tuple{ \text{ Simplification algébrique. } }$$
$$ = P(n) \hspace{2cm} \tuple{ \text{ définition du prédicat } }$$

Puisque $n$ a été choisi arbitrairement, nous pouvons conclure que $(\forall n \in \mathbb{N}^{*} \mid P(n-1) \implies P(n))$ et 

donc par le principe de l'induction mathématique $\frac{27}{4}3^n - \frac{1}{2}n - \frac{11}{4}$ est le terme générale de la suite $\tuple{b_n}_{n \in \mathbb{N}}$.

\vspace{0.5cm}

$\hfill \square$

\end{document}
