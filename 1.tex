\documentclass{article}
\usepackage{amsmath}
\usepackage{amssymb}
\usepackage[margin=.5in]{geometry}
\usepackage{array} 
\newcommand{\tuple}[1]{\ensuremath{\left\langle #1 \right\rangle}}
\newcommand{\pare}[1]{ \left( #1 \right) }

\begin{document}
    
\textbf{Exercice 1}

\vspace{0.5cm}

\textbf{Démonstration} Selon le théorème 1.5.6 pour deux ensembles dénombrables leur produit cartésien est aussi dénombrable.

Montrons tout d'abord la dénombrabilité de $\mathbb{Z}_{< 0}$ alors.

On sait que les entiers sont dénombrables par théorème 1.5.7. Donc, c'est vrai aussi que

$$|\mathbb{Z}| \leq |\mathbb{N}| \hspace{2 cm}
\tuple{ \text{ réflexivité de définition 1.5.11 } } $$


C’est clair que les entiers négatifs font partie des entiers donc par proposition 1.5.14 il y a une fonction injective de $\mathbb{Z}_{< 0}$ 

vers $\mathbb{Z}$. Donc,

$$
|\mathbb{Z}_{< 0}| \leq |\mathbb{Z}| \hspace{2 cm}
\tuple{ \text{ définition 1.5.11 } }
$$


Par la transitivé de définition 1.5.11 on peut conclure que

$$
|\mathbb{Z}_{< 0}| \leq |\mathbb{N}| \hspace{2 cm}
\tuple{ \text{ car } |\mathbb{Z}| \leq |\mathbb{N}| \land  |\mathbb{Z}_{< 0}| \leq |\mathbb{Z}| \hspace{.1cm} }
$$


Maintenant, $\mathbb{N}$ est la plus petite cardinalité, donc, $\mathbb{Z}_{< 0}$ comme ensemble infini on a

$$
|\mathbb{Z}_{< 0}| \geq |\mathbb{N}| \hspace{2 cm}
\tuple{ \text{ Théorème 1.5.15 } }
$$

Grâce à Bernstein-Schroeder ça implique

$$
|\mathbb{Z}_{< 0}| = |\mathbb{N}| \hspace{2 cm}
\tuple{\text{ car } |\mathbb{Z}_{<0}| \leq |\mathbb{N}| \land  |\mathbb{Z}_{<0} \geq |\mathbb{N}| \hspace{.1cm} }
$$

Donc, nous avons montré que $\mathbb{Z}_{< 0}$ est dénombrable et on sait que $\mathbb{N}$ est dénombrable. Ainsi leur produit cartésien est 

aussi dénombrable
$$
|\mathbb{Z}_{<0} \times \mathbb{N}| = |\mathbb{N}| \hspace{2 cm}
\tuple{ \text{ par Théorème 1.5.6 } }
$$

$\hfill \square$

\end{document}
